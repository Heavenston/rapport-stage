\subsection{Axes d'étude et de recherche choisis}

Mon stage portait sur le développement d'un serveur de notifications performant
et robuste. L'objectif principal était de fournir aux utilisateurs des
informations {\bf en temps réel}, ce qui a imposé des contraintes spécifiques
sur le choix des technologies et l'architecture du système. Plusieurs axes
d'étude ont été définis pour atteindre cet objectif.

Le premier enjeu était le choix des technologies. J'étais libre de choisir,
mais je devais considérer la {\bf dette technique} potentielle. Utiliser des
technologies peu maîtrisées par l'équipe aurait compliqué la maintenance. Il
fallait trouver un équilibre entre {\bf innovation et familiarité}.

De plus, la {\bf communication} entre le Backend, le Frontend et mon service
devait être simple. L'utilisation de technologies similaires aux services
existants permettrait de réutiliser les mêmes interfaces TypeScript et ainsi
réduire les risques d'incompatibilité.

Un autre axe important était de déterminer le {\bf protocole de communication}
entre les clients et le serveur de notifications. La nécessité d'une
communication en {\bf temps réel} a immédiatement réduit le champ des possibles.
Les requêtes HTTP traditionnelles, basées sur un modèle requête-réponse, ne
sont pas adaptées à ce besoin car elles obligeraient les clients à interroger
le serveur périodiquement (polling) pour vérifier la présence de nouvelles
notifications. Cette approche est inefficace en termes de consommation de
ressources (bande passante, CPU) et introduit une latence importante entre
l'émission d'une notification et sa réception par le client.

Pour garantir une communication en temps réel, il était nécessaire de se tourner
vers des technologies permettant une communication {\bf bidirectionnelle} ou
{\bf unidirectionnelle persistante} entre le serveur de notification et les
clients. Deux technologies principales se sont dégagées : {\bf WebSockets} et
{\bf Server-Sent Events (SSE)}.

\begin{itemize}
	\item {\bf WebSockets} établit une connexion bidirectionnelle et persistante
entre le client et le serveur.  Cela permet au serveur de "pousser" des
notifications vers le client dès qu'elles sont disponibles, sans que le client
n'ait à les demander.
	\item {\bf SSE}, quant à lui, permet une communication unidirectionnelle du
serveur vers le client via une connexion HTTP persistante. Le serveur peut ainsi
envoyer des données au client en temps réel.
\end{itemize}

Le choix entre ces deux technologies devait se faire selon la facilité
d'implémentation, la performance et la compatibilité avec l'architecture
existante, tout en gardant à l'esprit le besoin fondamental de {\bf temps réel}.
C'est-à-dire que quand un usager prend rendez-vous sur le site, il faut
qu'{\bf instantanément} le professionnel reçoive une notification pour le prévenir de
la prise de rendez-vous.

Enfin, le dernier axe de recherche était de concevoir une {\bf architecture
générique} pour le serveur de notifications. Le serveur devait être capable
de transmettre n'importe quel type de notification {\bf sans en interpréter le
contenu}. Cela posait la question de la structuration des données et de leur
gestion efficace. Comment garantir la flexibilité du système sans connaître la
structure interne des notifications ? Permettre au serveur de notifications de
lire les notifications poserait le problème de devoir lui permettre aussi de
lire la base de données principale, et donc partager la structure de celle-ci,
augmentant la difficulté de changement.

\subsection{Déroulement concret des études, expérimentations, mises au point...}

Les technologies que j'ai finalement utilisées sont :

\begin{itemize}
	\item {\bf Fastify}, à l'opposé de {\bf Express} utilisé pour le Backend. Ces technologies
gèrent les requêtes HTTP. La différence entre mon service et le Backend ne pose
pas de problèmes majeurs.
{\bf Fastify} est plus rapide qu'{\bf Express}, ce qui est crucial pour un serveur de
notification. De plus, il offre une meilleure validation des données, améliorant
la robustesse du système.
	\item {\bf MongoDB} comme base de données, la même que le Backend. Cela simplifie le
déploiement et la gestion des données.
\end{itemize}

Pour la connexion des clients, j'ai choisi les {\bf Server-Sent Events (SSE)}
plutôt que les {\bf WebSockets}. Ce choix a été motivé par :

\begin{itemize}
    \item {\bf Simplicité}: Les SSE sont nativement supportés par les navigateurs et simples
à implémenter côté serveur.
    \item {\bf Adaptation au besoin unidirectionnel}: Les notifications vont du serveur vers
les clients, les SSE sont donc parfaitement adaptés.
    \item {\bf Reconnexion automatique}: En cas de perte de connexion, les navigateurs se
reconnectent automatiquement au flux SSE.
\end{itemize}

\subsubsection{Transparence des notifications}

Le serveur de notification a été pensé depuis le début pour être complètement
indépendant des données sauvegardées à l'intérieur des notifications.

Cela a plusieurs avantages:

\begin{itemize}
	\item {\bf Découplage fort :} Évolutions indépendantes des services, pas de dépendance au format des notifications.
	\item {\bf Sécurité renforcée :} Accès limité aux données sensibles, réduit les risques.
	\item {\bf Flexibilité et évolutivité :} Gère tout type de notification, ajout facile de nouveaux types.
	\item {\bf Maintenance simplifiée :} Logique de traitement déportée, focus sur la transmission.
	\item {\bf Réutilisabilité :} Composant générique, utilisable dans d'autres projets.
	\item {\bf Indépendance de la base de données principale:} Pas d'accès, pas de dépendance à sa structure.
	\item {\bf Simplicité :} Code plus clair et facile à comprendre, focus sur l'essentiel.
\end{itemize}

Pour cela la plus grande difficulte vient de savoir quelle notification transmettre
a quel professionel. Comme le serveur de notification n'a pas access aux donnees de la notifications
il ne peut pas savoir a qui cette notfication est destinee.

Pour resoudre ce probleme, j'ai choisi un systeme peu conventionel:

Quand un professionnel se connecte à Allaw, le Front va faire une requête au
Backend, et celui-ci va lui donner un {\bf Token}. Ce token sera spécialement créé
par le Backend pour contenir un 'filtre', ce filtre décrit quelles notifications
il est autorisé à recevoir. C'est avec ce token qu'il pourra se connecter
directement au serveur de notifications, qui pourra, grâce à ce filtre dont il
ne connaît pas la logique pour le créer, savoir quelles notifications seront
renvoyées.

Ce token est constitué d'une {\bf expression} en JSON similaire aux {\bf query}
MongoDB, ce qui permet une grande flexibilité. Ces expressions que j'ai
surnommées des expressions {\bf Alang} sont un véritable langage de programmation
permettant d'exprimer de nombreuses fonctionnalités complètement dans ce token.

\subsubsection{Marquage comme lu des notifications}

Pour pouvoir savoir quelles notifications ont été vues par le professionnel,
un système similaire de token a été mis en place, permettant au front de
marquer comme vue n'importe quelle notification directement sur le serveur de
notifications.

Ces tokens contiennent une liste d'actions qui peuvent être prises, que le
front peut choisir, suivie d'une liste de notifications sur lesquelles effectuer
l'action.

\subsubsection{Gestion des données}

Le Backend fait une requete autentifier au serveur de notification pour publier
une nouvelle notifications, quand par exemple un nouveau rendez-vous a ete pris.

Le serveur de notifications fait passer par deux systèmes toutes les
notifications reçues de cette manière :

\begin{itemize}
	\item Un dont le but est de transférer les notifications vers toutes les
connexions actives.
	\item Un autre qui sauvegarde la notification dans la base de données pour pouvoir la
renvoyer aux pros qui n'étaient pas en ligne lors de la prise de rendez-vous.
\end{itemize}

Les notifications sont vérifiées envers tous les filtres de tous les
professionnels connectés actuellement, et sont envoyées vers ceux qui
correspondent. Puis elle est enregistrée dans la base de données, avec des
données supplémentaires aux données de la notification, comme la date d'envoi.
Cette enrengistrement permet de renvoyer la notifications a la connection d'un
professionel si jamais il ne l'avais pas vu avant.
