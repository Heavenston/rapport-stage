Allaw est une jeune startup nantaise créée il y a un an par un père et son fils,
dont le but est de révolutionner le monde du droit avec une ambition simple :
créer le Doctolib du droit.

% FIX: Language technique
Suite à une mise en relation par l'intermédiaire de contacts communs, j'ai pu
échanger avec Ludovic Stang, le CEO d'Allaw, en février 2024. Cet échange a
abouti, quelques mois plus tard, à une journée d'immersion dans l'entreprise
durant laquelle j'ai pu avoir un avant-goût de ce que je serais amené à faire
durant mon stage. J'ai pu également rencontrer une partie de l'équipe, qui s'est
depuis agrandie par la venue d'un développeur front ainsi que d'une nouvelle
"sale".

Mon intérêt pour cette entreprise vient de plusieurs points :
\begin{itemize}
  \item Je connais déjà très bien les technologies utilisées par l'entreprise (Node.js,
TypeScript, MongoDB), ce qui me permet de m'intégrer sans avoir à me préoccuper
de ne pas être à l'aise techniquement lors de la découverte du travail en
entreprise.

  \item Le fait que l'entreprise soit petite (10 personnes), jeune mais en pleine
expansion me permet de découvrir le travail en équipe de façon plus immersive
et d'avoir un impact significatif sur le développement du produit. Enfin, cela
permet d'appréhender une entreprise auprès de tous ses services, par exemple
j'interagis souvent avec l'équipe "sales", ainsi que d'être en relation directe
avec le management.

  \item L'entreprise évolue dans un secteur en pleine transformation numérique,
ce qui offre de nombreuses opportunités d'innovation et d'apprentissage.

  \item La mission d'Allaw, qui vise à simplifier la prise de rendez-vous pour les
particuliers, à démocratiser l'accès au droit et à simplifier le quotidien des
professionnels juridiques, correspond à mes valeurs personnelles.

  \item L'ambiance de travail jeune et dynamique, ainsi que la culture d'entreprise
orientée vers l'innovation et le partage des connaissances, correspondent
parfaitement à mes attentes pour une première expérience professionnelle.

  \item Enfin, le fait que l'entreprise soit située à Nantes était un vrai
avantage pour moi étant donné que je suis nantais.
\end{itemize}

Je suis donc au cœur de la dynamique d'une création d'entreprise où je découvre
toutes les étapes importantes pour son expansion (embauche, levées de fonds,
etc.). La parfaite intégration dans l'entreprise me permet de contribuer à la
création de ses fondations techniques et humaines.

Je ne regrette pas d'avoir intégré cette jeune entreprise dans laquelle je
me suis rapidement senti intégré, y compris sur le plan humain. En effet, les
stratégies mises en place pour la cohésion des équipes sont nombreuses (des
sorties, des jeux, etc.).
