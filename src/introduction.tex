Allaw est une startup nantaise dont le but est de révolutionner le monde du droit
avec une prémisse simple : créer le Doctolib du droit.

C'est grâce à des connaissances communes que j'ai pu prendre contact avec Ludovic Stang,
le CEO d'Allaw, en février 2024.

Mon intérêt pour cette entreprise vient de plusieurs points :
\begin{itemize}
  \item Je connais déjà très bien les technologies utilisées par l'entreprise
        (Node.js, TypeScript, MongoDB),
        ce qui me permet de m'intégrer sans avoir à me préoccuper de ne pas
        être à l'aise techniquement lors de la découverte du travail en entreprise.
  \item Le fait que l'entreprise soit petite (10 personnes) mais en pleine expansion
        me permet de découvrir le travail en équipe de façon plus naturelle et
        d'avoir un impact significatif sur le développement du produit.
  \item L'entreprise évolue dans un secteur en pleine transformation numérique,
        ce qui offre de nombreuses opportunités d'innovation et d'apprentissage.
  \item La mission d'Allaw, qui vise à démocratiser l'accès au droit et à
        simplifier le quotidien des professionnels juridiques, correspond à
        mes valeurs personnelles.
  \item L'ambiance de travail jeune et dynamique, ainsi que la culture d'entreprise
        orientée vers l'innovation et le partage des connaissances, correspondent
        parfaitement à mes attentes pour une première expérience professionnelle.
  \item Enfin, le fait que l'entreprise soit située à Nantes était un vrai
        avantage pour moi étant donné que je suis nantais.
\end{itemize}

