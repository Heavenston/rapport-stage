
\begin{itemize}
\item {\bf Alang} : Langage d'expressions JSON développé pour le serveur de notifications d'Allaw, permettant de filtrer les notifications à transmettre aux utilisateurs.
\item {\bf Backend} : Partie d'une application qui gère le traitement des données et la logique métier côté serveur.
\item {\bf Business Developer} : Professionnel chargé du développement commercial et de la croissance de l'entreprise.
\item {\bf Dette technique} : Conséquence de l'adoption de solutions techniques rapides mais imparfaites, nécessitant un travail supplémentaire futur.
\item {\bf DevOps} : Pratiques et outils combinant le développement logiciel (Dev) et l'administration système (Ops).
\item {\bf Doctolib} : Plateforme de prise de rendez-vous médicaux en ligne, servant de modèle à Allaw pour le secteur juridique.
\item {\bf Elasticsearch} : Moteur de recherche distribué, utilisé pour améliorer les capacités de recherche textuelle.
\item {\bf Express} : Framework web pour Node.js utilisé pour créer des applications web et des API.
\item {\bf Fastify} : Framework web Node.js alternatif à Express, optimisé pour la performance.
\item {\bf Frontend} : Partie d'une application visible et interactive pour l'utilisateur dans le navigateur.
\item {\bf JavaScript} : Langage de programmation utilisé principalement pour le développement web.
\item {\bf JSON} : Format de données textuelles utilisé pour l'échange d'informations.
\item {\bf Kafka} : Plateforme de traitement de flux de données en temps réel.
\item {\bf Legal Tech} : Technologies appliquées au domaine juridique pour moderniser et optimiser les services légaux.
\item {\bf Logique métier} : Règles et processus spécifiques au fonctionnement d'une entreprise ou d'un secteur d'activité.
\item {\bf MongoDB} : Base de données NoSQL orientée documents.
\item {\bf MVP (Minimum Viable Product)} : Version minimale d'un produit permettant de tester sa viabilité sur le marché.
\item {\bf Node.js} : Environnement d'exécution JavaScript côté serveur.
\item {\bf NoSQL} : Type de base de données non relationnelle, alternative aux bases de données SQL traditionnelles.
\item {\bf Polling} : Technique consistant à interroger régulièrement un serveur pour obtenir des mises à jour.
\item {\bf Scalabilité} : Capacité d'un système à s'adapter à une charge croissante.
\item {\bf Server-Sent Events (SSE)} : Technologie permettant à un serveur d'envoyer des mises à jour à un client en temps réel.
\item {\bf Service Support} : Équipe chargée de l'assistance aux utilisateurs.
\item {\bf Token} : Jeton d'authentification ou d'autorisation permettant l'accès sécurisé à des ressources.
\item {\bf TypeScript} : Extension de JavaScript ajoutant le typage statique.
\item {\bf WebSocket} : Protocole de communication bidirectionnelle en temps réel entre un navigateur et un serveur.
\end{itemize}
