
Le développement du serveur de notifications a été une expérience riche en
enseignements, me confrontant à la complexité de la conception d'un système
temps réel performant et robuste.

\subsection{Bilan des choix technologiques et architecturaux}

Les choix technologiques et architecturaux effectués ont permis d'atteindre les
objectifs initiaux : fournir aux utilisateurs des informations en temps réel
de manière transparente, sans que le serveur n'ait à interpréter le contenu
des notifications.

\begin{itemize}
\item {\bf Technologies utilisées: }
	\begin{itemize}
	    \item {\bf Fastify: } Gestion des requêtes HTTP. Sa rapidité et sa robustesse,
en comparaison avec Express, ont contribué à la performance du serveur. La
validation des données intégrée a renforcé la fiabilité du système.
	    \item {\bf MongoDB: } Base de données NoSQL. Son utilisation, en cohérence avec le
reste de l'infrastructure, a simplifié le déploiement et la gestion des données.
	    \item {\bf Server-Sent Events (SSE): } Protocole de communication unidirectionnelle. Ce
choix s'est avéré judicieux pour la diffusion efficace des notifications et la
reconnexion automatique en cas de perte de connexion.
	\end{itemize}
\item {\bf Architecture: }
	\begin{itemize}
    \item {\bf Système de filtres (expressions Alang): } Intégré dans des tokens
générés par le Backend, il a permis une distribution ciblée des notifications
sans exposer la {\bf logique métier} au serveur.
		\begin{itemize}
        \item {\bf Flexibilité et performance : } Gestion d'une grande variété de
scénarios de notification.
        \item {\bf Simplicité pour le serveur : } Pas d'interprétation du contenu
des notifications.
		\end{itemize}
    \item {\bf Marquage des notifications comme lues: } Utilisation de tokens
pour une gestion efficace de l'état des notifications côté client, tout en
préservant l'indépendance du serveur.
	\end{itemize}
\end{itemize}

\subsubsection{Remise en question : {\it From scratch} vs. Solutions Cloud}

Le développement {\it from scratch} a été formateur mais a représenté un investissement conséquent. La question se pose : {\bf une solution cloud existante n'aurait-elle pas été plus judicieuse ?}

\subsubsection{Analyse des solutions Cloud alternatives}

De nombreux services managés de gestion de notifications existent :

\begin{itemize}
\item {\bf Exemples : } Firebase Cloud Messaging (FCM), Amazon SNS, Azure
Notification Hubs.
\item {\bf Fonctionnalités : } Infrastructures scalables, gestion de gros
volumes, différents protocoles (WebSockets, SSE...), fonctionnalités avancées
(segments d'utilisateurs, personnalisation, analyses), haute disponibilité et
sécurité.
\end{itemize}

\subsubsection{Avantages potentiels d'une solution Cloud}

\begin{itemize}
\item {\bf Concentration sur le métier : } Se focalise sur les aspects fonctionnels.
\item {\bf Temps de développement réduit : } Mise sur le marché plus rapide.
\item {\bf Coûts potentiellement réduits : } Facturation à l'usage, avantageux
au démarrage.
\item {\bf Gestion de l'infrastructure déléguée : } Sécurité, scalabilité et
haute disponibilité gérées par le fournisseur.
\end{itemize}

\subsubsection{Avantages du développement {\it from scratch}}

\begin{itemize}
\item {\bf Personnalisation totale : } Adaptation précise aux besoins.
\item {\bf Maîtrise complète : } Contrôle de l'architecture et du code.
\item {\bf Flexibilité accrue : } Ajout de fonctionnalités et optimisation facilités.
\item {\bf Expertise interne : } Développement de compétences précieuses et différenciantes.
\item {\bf Indépendance technologique : } Pas de verrouillage fournisseur.
\end{itemize}

\subsubsection{Inconvénients du développement {\it from scratch} et dette technique}

\begin{itemize}
\item {\bf Maintenance et évolution : } Dépendent de l'expertise de l'équipe,
risque en cas de départ.
\item {\bf Responsabilité accrue : } Charge de travail pour la sécurité, la
scalabilité et la haute disponibilité.
\item {\bf Coût de développement élevé : } Investissement en temps et en
ressources humaines conséquent. Le calcul du coût par rapport a une solution
cloud n'est pas certain d'être avantageux.
\end{itemize}

\subsubsection{Conclusion : Un choix stratégique à pondérer}

Le choix dépend des priorités du projet, des ressources et de la stratégie à
long terme.

\begin{itemize}
\item {\bf Solution Cloud : } Gain de temps, réduction de la dette technique,
concentration sur le métier.
\item {\bf  {\it From scratch} : } Maîtrise totale, personnalisation,
développement de compétences.
\end{itemize}
