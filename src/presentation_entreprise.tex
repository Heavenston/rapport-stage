\section{Le secteur d'activité}
Le secteur d'activité d'Allaw se situe à l'intersection du numérique et du droit, 
dans le domaine du "Legal Tech". Ce secteur est en pleine expansion, porté par la 
transformation numérique des professions juridiques.
Le marché français des services juridiques représente un potentiel important,
avec plus de 70 000 avocats et des milliers d'autres professionnels du droit
(notaires, huissiers, etc.).

\section{L'entreprise}
Allaw est une startup nantaise fondée en 2023 par un duo père-fils. L'entreprise 
compte actuellement une dizaine de collaborateurs et connaît une phase de croissance 
active. Sa mission principale est de simplifier la mise en relation entre les 
professionnels du droit et leurs clients, en s'inspirant du modèle de Doctolib 
dans le secteur médical.

\section{Le service}
Le service développé par Allaw est une plateforme numérique qui permet :
\begin{itemize}
    \item Aux particuliers de prendre facilement rendez-vous avec des professionnels du droit
    \item Aux professionnels juridiques de gérer efficacement leur agenda et leur clientèle
    \item D'automatiser certaines tâches administratives pour optimiser le temps des professionnels
    \item De faciliter la communication entre les clients et les professionnels du droit
\end{itemize}

\section{Le positionnement du stage dans les travaux de l'entreprise}
En tant que stagiaire au sein de l'équipe backend, ma mission s'inscrit dans le 
développement technique de la plateforme. Je travaille principalement sur :
\begin{itemize}
    \item Le développement de nouvelles fonctionnalités pour le serveur
    \item La création d'un système de notifications
    \item L'amélioration des performances et de la scalabilité de l'application
\end{itemize}

Ce stage contribue directement à l'évolution du produit dans une phase critique 
de croissance de l'entreprise.

