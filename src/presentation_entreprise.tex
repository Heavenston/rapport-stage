\section{Le secteur d'activité}
Le secteur d'activité d'Allaw se situe à l'intersection du numérique et du
droit, dans le domaine de la "Legal Tech". Allaw propose son service aux
notaires, aux avocats et aux commissaires de justice. Ce secteur est en pleine
expansion, porté par la transformation numérique des professions juridiques.
Le marché français des services juridiques représente un potentiel important,
composé de plus de 70 000 avocats, 6 000 notaires et 3 000 commissaires de
justice.

\subsection{La concurrence}
Allaw opère dans un secteur où plusieurs concurrents sont présents, chacun
couvrant une partie de ce domaine d'activité. Cependant, Allaw se distingue
par sa capacité unique à offrir une gestion complète des prises de rendez-vous.
Contrairement à d'autres qui se concentrent exclusivement sur la mise en
relation entre professionnels et clients ou sur la gestion administrative, Allaw
ambitionne d'intégrer ces deux aspects. Cette intégration permet non seulement
de faciliter la gestion des interactions, mais aussi d'améliorer l'expérience
utilisateur en offrant une interface harmonieuse et cohérente.

% En plus de cela, Allaw prend en compte les spécificités des différentes
% professions du droit, offrant des fonctionnalités adaptées aux besoins
% particuliers des avocats, notaires, et commissaires de justice. Cette
% personnalisation est un atout majeur qui renforce l'attrait de la plateforme
% pour les professionnels cherchant à rationaliser leurs processus tout en gardant
% une qualité de service élevée.

\section{L'entreprise}

\subsection{L'équipe}

% \begin{forest}
%   for tree={
%     draw,
%     align=center
%   },
%   forked edges,
%   [Ludovic Stang\\CEO
%     [Germain Stang\\CTO
%       [Nicolas Cathala\\Developeur Frontend
%       ]
%       [Vincent Desbrosses\\Developeur Frontend
%       ]
%       [Malo Legendre\\Stagiaire Backend
%       ]
%       [Titouan Goubet\\Alternant Frontend
%       ]
%       [Nawel Alami\\Alternante Communication
%       ]
%     ]
%   ]
%   \node [draw, fit=(current bounding box.south east) (current bounding box.north west)] {};
% \end{forest}

\tikzset{
  basic/.style  = {draw, drop shadow, font=\sffamily, rectangle, text width=, minimum width=4cm, align=center}, % align=center
  root/.style   = {basic, thin, fill=gray!45},
  level 2/.style = {basic, thin, fill=gray!30, text width=8em},
  level 3/.style = {basic, thin, fill=gray!20, text width=, minimum width=5cm, node distance=40pt}
}

\begin{tikzpicture}[
  level 1/.style={sibling distance=70mm},
  edge from parent/.style={->,draw},
  >=latex]

% Root
\node[root] {Ludovic Stang \\ \footnotesize CEO}
  % Level 1
  child {node[level 2] (c1) {Germain Stang \\ \footnotesize CTO}}
  child {node[level 2] (c2) {Cédric Béchu \\ \footnotesize Head Of Sales}};

% Level 2
\begin{scope}[every node/.style={level 3}]
\node [below of = c1, xshift=25pt] (c11) {Nicolas Cathala \\ \footnotesize Developeur Backend};
\node [below of = c11] (c12) {Vincent Desbrosses \\ \footnotesize Developeur Frontend};
\node [below of = c12] (c13) {Malo Legendre \\ \footnotesize Stagiaire Backend};
\node [below of = c13] (c14) {Titouan Goubet \\ \footnotesize Alternant Frontend};
\node [below of = c14] (c15) {Nawel Alami \\ \footnotesize Alternante Communication};
\node [below of = c2, xshift=25pt] (c21) {Yanis Zagrar \\ \footnotesize Business Developer};
\node [below of = c21] (c22) {Sarah Quintin \\ \footnotesize Support Client};
\end{scope}

% Lines
\foreach \value in {1,2,3,4,5}
  \draw[->] (c1.195) |- (c1\value.west);
\foreach \value in {1,2}
  \draw[->] (c2.195) |- (c2\value.west);
\end{tikzpicture}

\subsection{Presentation}

Allaw est une startup nantaise fondée en 2023 par un duo père-fils, Ludovic
Stang et son fils Germain. L'entreprise compte actuellement une dizaine de
collaborateurs et connaît une phase de croissance active. Sa mission principale
est de simplifier la mise en relation entre les professionnels du droit et leurs
clients, en s'inspirant du modèle de Doctolib dans le secteur médical. Doctolib
disposait d'une solide force de vente, un modèle qu'Allaw s'efforce d'imiter.

\section{Le service}
Le service développé par Allaw permet :
\begin{itemize}
    \item Aux particuliers de prendre facilement rendez-vous avec des professionnels du droit
    \item Aux professionnels juridiques de gérer efficacement leur agenda et leur clientèle
    \item D'automatiser certaines tâches administratives pour optimiser le temps des professionnels
    \item De faciliter la communication entre les clients et les professionnels du droit
\end{itemize}

Elle parvient à cela à travers les fonctionnalités suivantes :

\begin{itemize}
    \item Un référencement de tous les professionnels en France, y compris ceux non inscrits sur Allaw.
    \item Les agendas Outlook et Google directement synchronisés sur Allaw permettant de proposer aux particuliers des créneaux de disponibilités.
    \item Les visioconférences directement prises sur Allaw.
    \item Le paiement des prestations directement sur Allaw.
    \item Et bien d'autres...
\end{itemize}

Voir la page d'accueil d'Allaw dans l'annexe \ref{appendix:landing_page}.

\section{Le positionnement du stage dans les travaux de l'entreprise} 
En tant que stagiaire au sein de l'équipe backend, ma mission s'inscrit dans le développement technique de la plateforme. Je travaille principalement sur :
\begin{itemize}
    \item Le développement de nouvelles fonctionnalités pour le serveur
    \item La création d'un système de notifications
    \item L'amélioration des performances et de la scalabilité de l'application
\end{itemize}
Ce stage contribue directement à l'évolution du produit dans une phase critique
de croissance de l'entreprise.

L'objectif est de construire un produit robuste capable de supporter une
augmentation significative du nombre d'utilisateurs tout en maintenant la
stabilite requise au bon fonctionnement. Cela implique également de collaborer
étroitement avec les équipes de développement frontend pour assurer une
intégration fluide des nouvelles fonctionnalités.

% \todo[inline]{SUPRIMER?}
% Ma participation à ce projet me permet de développer de nouvelles compétences
% techniques clés tout en travaillant ma créativité en trouvant de nouvelles
% idées.
