\section{Le secteur d'activité} 
Le secteur d'activité d'Allaw se situe à l'intersection du numérique et du droit, dans le domaine de la "Legal Tech". Allaw propose son service aux notaires, aux avocats et aux commissaires de justice. Ce secteur est en pleine expansion, porté par la transformation numérique des professions juridiques. Le marché français des services juridiques représente un potentiel important, composé de plus de 70 000 avocats, 6 000 notaires et 3 000 commissaires de justice. 

\subsection{La concurrence} 
Allaw opère dans un secteur où plusieurs concurrents sont présents, chacun couvrant une partie de ce domaine d'activité. Cependant, Allaw se distingue par sa capacité unique à offrir une gestion complète des prises de rendez-vous. Contrairement à d'autres qui se concentrent exclusivement sur la mise en relation entre professionnels et clients ou sur la gestion administrative, Allaw ambitionne d'intégrer ces deux aspects. Cette intégration permet non seulement de faciliter la gestion des interactions, mais aussi d'améliorer l'expérience utilisateur en offrant une interface harmonieuse et cohérente.

En plus de cela, Allaw prend en compte les spécificités des différentes professions du droit, offrant des fonctionnalités adaptées aux besoins particuliers des avocats, notaires, et commissaires de justice. Cette personnalisation est un atout majeur qui renforce l'attrait de la plateforme pour les professionnels cherchant à rationaliser leurs processus tout en gardant une qualité de service élevée.

\section{L'entreprise} 
Allaw est une startup nantaise fondée en 2023 par un duo père-fils. L'entreprise compte actuellement une dizaine de collaborateurs et connaît une phase de croissance active. Sa mission principale est de simplifier la mise en relation entre les professionnels du droit et leurs clients, en s'inspirant du modèle de Doctolib dans le secteur médical. Doctolib disposait d'une solide force de vente, un modèle qu'Allaw s'efforce d'imiter, et qui est maintenant composée de 5 commerciaux contre 3 développeurs.

La jeunesse de l'entreprise lui confère une agilité notable, permettant de s'adapter rapidement aux évolutions du marché. En cultivant une culture d'entreprise dynamique et innovante, Allaw encourage ses employés à contribuer activement à l'amélioration continue de ses services. L'objectif principal est de créer un environnement de travail propice à l'épanouissement personnel et professionnel de chaque membre de l'équipe.

\section{Le service} 
Le service développé par Allaw est une plateforme numérique qui permet :
\begin{itemize}
    \item Aux particuliers de prendre facilement rendez-vous avec des professionnels du droit
    \item Aux professionnels juridiques de gérer efficacement leur agenda et leur clientèle
    \item D'automatiser certaines tâches administratives pour optimiser le temps des professionnels
    \item De faciliter la communication entre les clients et les professionnels du droit
\end{itemize}

En plus des fonctionnalités de base, Allaw prévoit d'introduire de nouvelles fonctionnalités telles que la gestion électronique des documents et la possibilité d'effectuer des consultations en ligne. Ces innovations visent à accroître l'efficacité des interactions juridiques et à élargir l'accessibilité aux services juridiques.

Une attention particulière est également accordée à la sécurité des informations partagées sur la plateforme. Allaw utilise des protocoles de sécurité avancés pour garantir la confidentialité et l'intégrité des données, un aspect crucial pour la confiance des utilisateurs.

\section{Le positionnement du stage dans les travaux de l'entreprise} 
En tant que stagiaire au sein de l'équipe backend, ma mission s'inscrit dans le développement technique de la plateforme. Je travaille principalement sur :
\begin{itemize}
    \item Le développement de nouvelles fonctionnalités pour le serveur
    \item La création d'un système de notifications
    \item L'amélioration des performances et de la scalabilité de l'application
\end{itemize}
Ce stage contribue directement à l'évolution du produit dans une phase critique de croissance de l'entreprise.

L'objectif est de construire un produit robuste capable de supporter une augmentation significative du nombre d'utilisateurs tout en maintenant une haute performance. Cela implique également de collaborer étroitement avec les équipes de développement frontend pour assurer une intégration fluide des nouvelles fonctionnalités.

De plus, ma participation à ce projet me permet de développer des compétences techniques clés en travaillant avec des technologies modernes. En tant que stagiaire, je suis encouragé à proposer des idées et des solutions innovantes, ce qui constitue une expérience d'apprentissage inestimable dans le monde professionnel.
