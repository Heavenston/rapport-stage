
\subsection{But général}

Au sein d'Allaw, mon but principal était d'aider au développement du backend
et donc d'aider Nicolas, l'unique développeur backend, dans ses tâches. Mon
sujet principal pour le stage était un serveur de notifications, recevant les
événements du backend et les redistribuant vers les clients connectés.

Mais j'ai aidé au développement de beaucoup d'autres fonctionnalités :

\begin{enumerate}
    \item Relier tous les professionnels inscrits et non-inscrits de manière aléatoire sur
          leurs pages pour aider le référencement Google. (le "maillage")
	\item La connexion des usagers passant par Google ou LinkedIn. Voir annexe \ref{appendix:signup}.
    \item La recherche des professionnels sur le site, paginée et aléatoire. Voir annexe \ref{appendix:research}.
	\item Les notifications grâce au serveur de notifications. Voir annexe \ref{appendix:notifications}.
	\item Régler la gestion des décalages horaires sur Allaw permettant d'ouvrir l'application aux outre-mer.
	\item Une barre de recherche permettant de rechercher parmi tous les objets de manière naturelle.
\end{enumerate}

À noter que pour chacune de ces fonctionnalités, le développement sur le {\bf
Frontend} a été fait par d'autres collègues, je me focalisais sur le {\bf
Backend}.

Je vais concentrer ce rapport sur le serveur de notifications, comme c'était mon
sujet de stage, ainsi que les décalages horaires et enfin la barre de recherche.

\subsection{Explication détaillée des résultats à obtenir}

\subsubsection{Le "maillage"}

À mon arrivée, l'équipe était en pleine refonte du site, pour améliorer leur
référencement Google. Dans ces travaux, se trouvait le besoin d'afficher en
dessous de tous les professionnels, une sélection de 10 autres professionnels
se trouvant dans la même ville. Et ceci de manière statique, pour que les
professionnels suggérés ne changent pas afin d'aider Google à référencer les
pages.

Ce "maillage" devait être suffisamment interconnecté pour permettre à tous les
professionnels d'être trouvés par Google en passant par un autre professionnel.

\subsubsection{Le serveur de notification}

Le serveur de notifications devait recevoir des événements du backend, comme
un nouveau rendez-vous, un paiement effectué, etc. Puis d'une manière ou d'une
autre les renvoyer à tous les clients en temps réel sur leur frontend.

À part ces conditions, j'étais complètement libre sur l'implémentation et les
choix de technologies.

Allaw ne disposant pas de DevOps à temps plein, il était difficile pour Nicolas,
le Backend, de trouver le temps de se former et de créer les services difficiles
à mettre en place tels que Kafka pour la communication inter-services. Pour
cette raison, le serveur de notifications n'aurait qu'une seule instance
pour toute l'application, ce qui augmente les charges sur un seul serveur,
contrairement au Backend qui possède trois instances. Il fallait donc un service
performant et robuste.

Ce serveur de notifications permettra aux professionnels gardant Allaw ouvert
dans la journée de savoir en temps réel quand un rendez-vous a été pris sans
avoir à actualiser sans cesse l'application. Voir annexe \ref{appendix:notifications}.

\subsubsection{Les decalages horaires}

Durant la phase de MVP d'Allaw, un support robuste des différents décalages
horaires sur la plateforme n'avait pas été réfléchi. En effet, les différentes
librairies de l'écosystème JavaScript gèrent de manière différente les
décalages horaires, et en essayant de tout accommoder, les dates étaient souvent
sauvegardées et transformées dans des formats non-standards.

Mon but était donc d'uniformiser cette gestion, permettant à quelqu'un
d'outre-mer de prendre rendez-vous avec un avocat se trouvant en métropole et
que les bonnes dates s'affichent.

Cette gestion permettra à Allaw de commencer à débaucher des professionnels
dans les DOM-TOM directement, ce qui étend largement leur secteur d'activité.
De plus, les professionnels du droit en DOM-TOM ont souvent des difficultés dans
la mise en relation avec leurs clients, Allaw est donc parfaitement positionné à
ce niveau-là.

\subsubsection{La barre de recherche global}

La base de données utilisée par Allaw est MongoDB. Or, les capacités de
recherche textuelle de cette base de données sont très simplistes et pas très
performantes. Il fallait donc mettre en place Elasticsearch qui offre des
capacités de recherche textuelle bien supérieures.

En plus de ce nouveau service, la recherche devait permettre de trouver des
rendez-vous grâce à une recherche naturelle. Par exemple, entrer "rendez-vous
avec Malo Legendre le 5 janvier 2025" devrait retourner tous les rendez-vous
avec "Malo Legendre" qui ont eu lieu le 5 janvier 2025.

Une autre fonctionnalité était de proposer de créer un rendez-vous, une note ou
autre objet si ceux-ci n'existent pas déjà.
